\documentclass[11pt, ngerman, fleqn, DIV=15, headinclude]{scrartcl}

\usepackage[bibatend, color]{../header}

\hypersetup{
    pdftitle=
}

\renewcommand{\thesubsection}{\thesection.\alph{subsection}}

\usepackage{listings}
\usepackage{beramono}
\lstset{
    basicstyle=\small\tt
}

%\subject{}
\title{Quantenmechanik, Blatt 6}
%\subtitle{}
\author{
    Frederike Schrödel \and Heike Herr \and Jan Weber \and Simon Schlepphorst
}


\begin{document}

\maketitle

\section{Operatoren - Formalismus}

\section{Projektoren}

\section{Matrixdarstellung von Operatoren}

\subsection{ }
Die gesuchte Matrix ist:
\begin{align*}
	\hat A = \begin{pmatrix}
		5 & \alpha & 0\\
		\beta & 0 & \iup\\
		0 & -\iup & \gamma
	\end{pmatrix}
\end{align*}

\subsection{ }
Damit $\hat A$ hermitesch ist muss gelten:
\begin{align*}
	\hat A = \hat A^\dagger
\end{align*}
Das ist gegeben für $\gamma \in \mathbb R$ und $\alpha, \beta \in \mathbb C$
mit $\alpha^* = \beta$.

\subsection{ }
Im folgenden wird $\alpha = \beta = 0$ gesetzt und die Matrix $\hat A$ genutzt.
\begin{align*}
	\hat A = \begin{pmatrix}
		5 & 0 & 0\\
		0 & 0 & \iup\\
		0 & -\iup & \gamma
	\end{pmatrix}
\end{align*}

Es sind die Eigenvektoren zu bestimmen.
\begin{align*}
	\det\del{\hat A - \lambda \mathbb 1} &=
	\begin{vmatrix}
		5-\lambda & 0 & 0\\
		0 & -\lambda & \iup\\
		0 & -\iup & \gamma-\lambda
	\end{vmatrix}\\
	&= \del{\lambda - 5} \del{\lambda - \del{\frac\gamma2 + \frac12
		\sqrt{\gamma^2 + 4}}} \del{\lambda - \del{\frac\gamma2 - \frac12
			\sqrt{\gamma^2 + 4}}}
			\overset{!}{=} 0
	\intertext{Damit ergeben sich die normierten Eigenvektoren}
	\vec v_1 &= \begin{pmatrix}
		1\\
		0\\
		0
	\end{pmatrix}\\
	\vec v_2 &= \frac1{\sqrt{\frac12 \gamma\sqrt{\gamma^2 + 4}}}
	\begin{pmatrix}
		0\\
		-\frac\iup2\gamma + \frac\iup2 \sqrt{\gamma^2 + 4}\\
		1
	\end{pmatrix}\\
	\vec v_3 &= \frac1{\sqrt{\frac12\del{\gamma\sqrt{\gamma^2 + 4} +
\gamma^2 + 4}}}
	\begin{pmatrix}
		0\\
		-\frac\iup2\gamma - \frac\iup2 \sqrt{\gamma^2 + 4}\\
		1
	\end{pmatrix}
\end{align*}

\subsection{ }
\begin{align*}
	\Bracket{\hat A} &= \Braket{\psi^* | \hat A | \psi} = \begin{pmatrix}
		a^* & b^* & 0
	\end{pmatrix} \begin{pmatrix}
		5 & 0 & 0\\
		0 & 0 & \iup\\
		0 & -\iup & \gamma
	\end{pmatrix} \begin{pmatrix}
		a \\
		b \\
		0
	\end{pmatrix} \\
	&= 5\abs{a}^2
\end{align*}

\subsection{ }
\fehlt %TODO 

\subsection{ }
\fehlt %TODO


\section{Zwei-Niveau System}



\end{document}

% vim: spell spelllang=de tw=79
