\documentclass[11pt, ngerman, fleqn, DIV=15, headinclude]{scrartcl}

\usepackage[bibatend, color]{../header}

\hypersetup{
    pdftitle=
}

\renewcommand{\thesubsection}{\thesection.\alph{subsection}}

\usepackage{units}
\usepackage{listings}
\usepackage{beramono}
\lstset{
    basicstyle=\small\tt
}
\newcommand{\norm}[1]{\left\lVert#1\right\rVert}

%\subject{}
\title{Quantenmechanik, Blatt 10}
%\subtitle{}
\author{
    Frederike Schrödel \and Heike Herr \and Jan Weber \and Simon Schlepphorst
}

\newcommand\odx[3]{\frac{\dif^{#1} {#2}}{\dif {#3}^{#1}}}

\usepackage{cancel}
\newcommand{\colorcancel}[2]{\renewcommand{\CancelColor}{\color{#2}}\cancel{#1}}



\begin{document}

\maketitle
\begin{center}
	\begin{tabular}{l|c|c|c|c}
		Aufgabe &1&2&3&$\Sigma$\\
		\hline
		Punkte &\quad /8 & \quad /11 & \quad /27 & \quad /46
	\end{tabular}\\
\end{center}

\section{Bahndrehimpuls eines Elektrons}

\subsection{}
	Die Eigenfunktionen von dem Bahndrehimpulsoperator $\hat{L}_z$ haben die Form $Y_{lm}(\vartheta,\varphi)\cdot g(r)$ wobei $Y_{lm}(\vartheta,\varphi)$ die Kugelflächenfunktion mit Parametern $l$ und $m$ ist, $g(r)$ eine Funktion.
	Wir wissen:
	\begin{align*}
		Y_{11}(\vartheta,\varphi)&= \sqrt{\frac{3}{8\pi}}\sin(\vartheta)\eup^{i\varphi} \\
		Y_{10}(\vartheta,\varphi)&=\sqrt{\frac{3}{4\pi}}\cos(\vartheta)
	\end{align*}
	Somit können wir den Zustand $\psi$ als Linearkombinationen aus den beiden Eigenfunktionen $Y_{11}(\vartheta,\varphi)g(r)$ zum Eigenwert $\hbar$ und $Y_{10}(\vartheta,\varphi)g(r)$ zum Eigenwert $0$ schreiben:
	\begin{equation*}
		\psi=\sqrt{\frac{2}{3}}\cdot Y_{11}(\vartheta,\varphi)g(r) + \frac{1}{\sqrt{3}}\cdot Y_{10}(\vartheta,\varphi)g(r)
	\end{equation*}
	Somit sind die möglichen Ergebnisse einer Messung $0$ und $\hbar$.
	
\subsection{}
	Die Wahrscheinlichkeit $0$ zu messen beträgt $\frac{1}{3}=|\langle\psi|Y_{10}\rangle|^2$, die Wahrscheinlichkeit $\hbar$ zu messen beträgt $\frac{2}{3}$.
	

\subsection{}
	Daraus ergibt sich als Erwartungswert $\langle\hat{L}_z\rangle$:
	\begin{equation*}
		\langle\hat{L}_z\rangle=0\cdot|\langle\psi|Y_{10}\rangle|^2+\hbar\cdot|\langle\psi|Y_{11}\rangle|^2=\frac{2}{3}\hbar
	\end{equation*}

\section{Das Zweiatomige Molekül: Rotation und Vibration}

Betrachtet wird folgende Schrödingergleichung:
\begin{align*}
	\del{-\frac{\hbar^2}{2\mu}\pdx{2}{}{r} + V\del{r} + V_c\del{r}}u_{n,r}\del{r} = E_{n,l}u_{n,l}\del{r}
\end{align*}
Dabei sind: $\mu = \frac{m_1m_2}{m_1 + m_2}$, $V_c\del{r} = \frac{l\del{l + 1}\hbar^2}{2\mu r^2}$\\

\subsection{}

Nun sei $l=0$. Es soll $V\del{r}$ quadratisch genähert werden.\\
Ich setzte:
\begin{align*}
	V\del{r} = \half b^2\del{r - r_b}^2 - V_0
\end{align*}
Mit $V_0 = konstant$. Damit folgt:
\begin{align*}
	\del{-\frac{\hbar^2}{2\mu}\pdx{2}{}{r} + \half b^2\del{r - r_b}^2}u_{n,r}\del{r} = \del{E_{n,l} + V_0}u_{n,l}\del{r}
\end{align*}
Dies entspricht der Differentialgleichung für den harmonischen Oszillator. Die Energiewerte sind wohlbekannt:
\begin{align*}
	E_n = \hbar b\del{n + \half} - V_0
\end{align*}

\subsection{}

Sei immer noch $l=0$. Sei $r' = r - r_b$. Es soll die charakteristische Ausdehnung $\del{\Delta r'}_0$ des Grundzustandes $\ket{0}$ berechnet werden:
\begin{align*}
	\del{\Delta r'}_0	&= \sqrt{\bracket{r'^2}_0 + \bracket{r'}^2_0}\\
						&= \sqrt{\bra{0}r'^2\ket{0} + \bra{0}r'\ket{0}^2}\\
	\intertext{%
		Sei $a\equiv\sqrt{\frac{\hbar}{\mu b}}$. Dann kann $r'$ durch die Leiteroperatoren wie folgt ausgedrückt werden: $r' = a\del{\hat{a}^{\dagger} + \hat{a}}$. Damit folgt:
	}
						&= \sqrt{\bra{0}a^2\del{\hat{a}^{\dagger} + \hat{a}}^2\ket{0} + \underbrace{\bra{0}a\del{\hat{a}^{\dagger} + \hat{a}}\ket{0}^2}_{=0}}\\
						&= \sqrt{\bra{0}a^2\del{\hat{a}^{\dagger^2} + \hat{a}^2 + \hat{a}^{\dagger}\hat{a} + \hat{a}\hat{a}^{\dagger}}\ket{0}}\\
						&= a\sqrt{\underbrace{\bra{0}\hat{a}^{\dagger^2}\ket{0}}_{=0} + \underbrace{\bra{0}\hat{a}^2\ket{0}}_{=0} + \bra{0}\underbrace{\hat{a}^{\dagger}\hat{a} + \hat{a}\hat{a}^{\dagger}}_{\hat{N} + 1}\ket{0}}\\
						&= a\sqrt{\underbrace{\bra{0}\hat{N}\ket{0}}_{=0} + \underbrace{\braket{0|0}}_{=1}}\\
						&= a
\end{align*}

\section{Störungen des Wasserstoffatoms}

Betrachtet wird der Hamiltonoperator des Wasserstoffatoms unter Einfluss eines zusätzlichen Störpotentials
\begin{align*}
	\hat{H}=-\frac{\hbar^2}{2m}\laplace - \frac{e^2}r + \frac{\tilde{\epsilon}}{r^2}
\end{align*}
mit $\tilde{\epsilon}$ konstant.

\subsection{}

Es soll gezeigt werden, dass sich für den radialen Anteil folgende Differentialgleichung ergibt:
\begin{align*}
	\del{-\frac{\hbar^2}{2m}\odx{2}{}{r} + \frac1{r^2}\frac{\hbar^2}{2m}\del{l\del{l + 1} + \epsilon} - \frac{e^2}r}R\del{r}=ER\del{r}
\end{align*}
Für $\hat{H}$ ergibt sich:
\begin{align*}
	\hat{H}	&= -\frac{\hbar^2}{2m}\laplace - \frac{e^2}r + \frac{\tilde{\epsilon}}{r^2}\\
			&= -\frac{\hbar^2}{2m}\del{\frac1{r^2}\pd{}{r}r^2\pd{}{r} - \frac{\hat{L}^2}{\hbar^2r^2}} - \frac{e^2}r + \frac{\tilde{\epsilon}}{r^2}
\end{align*}
Setze nun:
\begin{align*}
	\psi\del{r,\theta,\varphi} = Y\del{\theta,\varphi}\frac{R\del{r}}{r}
\end{align*}
Anwenden von $\hat{H}$ auf $\psi$ gibt dann:
\begin{align*}
	\hat{H}\psi\del{r,\theta,\varphi}	&= \del{-\frac{\hbar^2}{2m}\del{\frac1{r^2}\pd{}{r}r^2\pd{}{r} - \frac{\hat{L}^2}{\hbar^2r^2}} - \frac{e^2}r + \frac{\tilde{\epsilon}}{r^2}}Y\del{\theta,\varphi}\frac{R\del{r}}{r}\\
										&= \del{-\frac{\hbar^2}{2m}\del{\frac1{r^2}\pd{}{r}\del{rR'\del{r} - R\del{r}} - \frac{\hat{L}^2}{\hbar^2r^3}R\del{r}} - \frac{e^2}{r^2}R\del{r} + \frac{\tilde{\epsilon}R\del{r}}{r^3}}Y\del{\theta,\varphi}\\
										&= \del{-\frac{\hbar^2}{2m}\del{\frac{R''\del{r}}{r} - \frac{\hat{L}^2}{\hbar^2r^3}R\del{r}} - \frac{e^2}{r^2}R\del{r} + \frac{\tilde{\epsilon}R\del{r}}{r^3}}Y\del{\theta,\varphi}\\
										&= \del{-\frac{\hbar^2}{2m}\del{\frac{R''\del{r}}{r} - \frac{l\del{l + 1}}{r^3}R\del{r}} - \frac{e^2}{r^2}R\del{r} + \frac{\tilde{\epsilon}R\del{r}}{r^3}}Y\del{\theta,\varphi}
\end{align*}
Kürzen von $\frac{Y\del{\theta,\varphi}}{r}$ gibt dann:
\begin{align}
	&\del{-\frac{\hbar^2}{2m}\del{\odx{2}{}{r} - \frac{l\del{l + 1}}{r^2}} - \frac{e^2}r + \frac{\tilde{\epsilon}}{r^2}}R\del{r} = ER\del{r}\nonumber\\
	\Leftrightarrow&\del{-\frac{\hbar^2}{2m}\odx{2}{}{r} + \frac{\hbar^2l\del{l + 1}}{2mr^2} - \frac{e^2}r + \frac{\tilde{\epsilon}}{r^2}}R\del{r} = ER\del{r}\nonumber\\
	\Leftrightarrow&\del{-\frac{\hbar^2}{2m}\odx{2}{}{r} + \frac{\hbar^2\del{l\del{l + 1} + \frac{2m\tilde{\epsilon}}{\hbar^2}}}{2mr^2} - \frac{e^2}r}R\del{r} = ER\del{r}\nonumber\\
	\intertext{%
		Setze $\epsilon\equiv\frac{2m\tilde{\epsilon}}{\hbar^2}$
	}
	\Leftrightarrow&\del{-\frac{\hbar^2}{2m}\odx{2}{}{r} + \frac{\hbar^2\del{l\del{l + 1} + \epsilon}}{2mr^2} - \frac{e^2}r}R\del{r} = ER\del{r}
\end{align}

%Setzte dazu erst mal $\alpha\equiv\frac{\hbar^2}{2m}$ und $\beta\equiv\del{l\del{l + 1} + \epsilon}$.

\subsection{}

Nun soll das Verhalten für $r\to0$ diskutiert werden.\\
Für $r\to0$ dominiert der quadratisch in $r$ abfallende Term, so dass folgt:
\begin{align*}
	\odx{2}{}{r}R\del{r} = \frac{l\del{l + 1} + \epsilon}{r^2}R\del{r}
\end{align*}
Setze $R\del{r}\propto r^{\sigma}$:
\begin{align*}
	&\odx{2}{}{r}r^{\sigma} = \frac{l\del{l + 1} + \epsilon}{r^2}r^{\sigma}\\
	\Leftrightarrow&\sigma\del{\sigma + 1}r^{\sigma - 2} = \del{l\del{l + 1} + \epsilon}r^{\sigma - 2}\\
	\Leftrightarrow&\sigma\del{\sigma + 1} = \del{l\del{l + 1} + \epsilon}\\
	\Rightarrow&\sigma = \half \pm \sqrt{\frac14 + l\del{l + 1} + \epsilon}\\
	\Rightarrow&R\del{r}\propto r^{\half + \sqrt{\frac14 + l\del{l + 1} + \epsilon}}
\end{align*}
Die negative Lösung für $\sigma$ fällt weg, da wir Normierbarkeit fordern.

\subsection{}

Nun soll der andere Grenzfall für $r\to\infty$ diskutiert werden.\\
Für $r\to\infty$ dominiert der lineare Term $ER\del{r}$ gegenüber den mit $r$ abfallenden, so dass folgt:
\begin{align*}
	&\odx{2}{}{r}R\del{r} = -\frac{2mE}{\hbar^2}R\del{r}
\end{align*}
Betrachtet werden gebundene Zustände, also ist die Energie negativ. Setze darum $\kappa\equiv\sqrt{-\frac{2mE}{\hbar^2}}$:
\begin{align*}
	&\odx{2}{}{r}R\del{r} = \kappa^2R\del{r}
\end{align*}
Es ergibt sich also für $R\del{r}$ 
\begin{align*}
	R\del{r} \propto \eup^{-\kappa r} + \eup^{\kappa r}
\end{align*}
Wiederum folgt aus der Forderung, dass $R\del{r}$ normierbar sein soll:
\begin{align*}
	R\del{r} \propto \eup^{-\kappa r}\\
\end{align*}

Wir bekommen also:
\begin{align*}
	R\del{r} = r^{\sigma}\eup^{-\kappa r}W\del{r}
\end{align*}

\subsection{}

Nun soll für $W\del{r}$ eine Differentialgleichung hergeleitet werden.\\
Dazu betrachte ich $\hat{H}R\del{r}$. Da das recht unübersichtlich wird, betrachte ich zuerst $R''\del{r}$:
\begin{align*}
	R''\del{r}	&= \odx{2}{}{r}\del{r^{\sigma}\eup^{-\kappa r}W\del{r}}\\
				&= \od{}{r}\del{\sigma r^{\sigma - 1}\eup^{-\kappa r}W\del{r} - \kappa r^{\sigma}\eup^{-\kappa r}W\del{r} + r^{\sigma}\eup^{-\kappa r}W'\del{r}}\\
				&= \sigma\del{\sigma - 1}r^{\sigma - 2}\eup^{-\kappa r}W\del{r} - \kappa\sigma r^{\sigma - 1}\eup^{-\kappa r}W\del{r} + \sigma r^{\sigma - 1}\eup^{-\kappa r}W'\del{r}\\
				&\;\;\;\; - \sigma\kappa r^{\sigma - 1}\eup^{-\kappa r}W\del{r} + \kappa^2r^{\sigma}\eup^{-\kappa r}W\del{r} - \kappa r^{\sigma}\eup^{-\kappa r}W'\del{r}\\
				&\;\;\;\; + \sigma r^{\sigma - 1}\eup^{-\kappa r}W'\del{r} - \kappa r^{\sigma}\eup^{-\kappa r}W'\del{r} + r^{\sigma}\eup^{-\kappa r}W''\del{r}
\end{align*}
Es folgt:
\begin{align*}
	R''\del{r}r^{-\sigma}\eup^{\kappa r} &= \frac1{r^2}\sigma\del{\sigma - 1}W\del{r} - \frac1r\kappa\sigma W\del{r} + \frac1r\sigma W'\del{r}\\
				&\;\;\;\; - \frac1r\sigma\kappa W\del{r} + \kappa^2W\del{r} - \kappa W'\del{r}\\
				&\;\;\;\; + \frac1r\sigma W'\del{r} - \kappa W'\del{r} + W''\del{r}\\
				&= \del{\frac{\sigma\del{\sigma - 1}}{r^2} - \frac{2\kappa\sigma}r + \kappa^2}W\del{r} +2 \del{\frac{\sigma}r - \kappa}W'\del{r} + W''\del{r}
\end{align*}
Setze $\lambda\equiv\frac{2me^2}{\hbar^2}$. Damit kann (1) wie folgt geschrieben werden:
\begin{align*}
	&\del{-\odx{2}{}{r} + \frac{\sigma\del{\sigma - 1}}{r^2} - \frac{\lambda}r}R\del{r} = -\kappa^2R\del{r}\\
	\Leftrightarrow& -R''\del{r} + \del{\frac{\sigma\del{\sigma - 1}}{r^2} - \frac{\lambda}r}R\del{r} = -\kappa^2R\del{r}\\
	\Leftrightarrow& -\del{\colorcancel{\frac{\sigma\del{\sigma - 1}}{r^2}}{blue} - \frac{2\kappa\sigma}r + \colorcancel{\kappa^2}{red}}W\del{r} - 2\del{\frac{\sigma}r - \kappa}W'\del{r} - W''\del{r}\\
	&+ \del{\colorcancel{\frac{\sigma\del{\sigma - 1}}{r^2}}{blue} - \frac{\lambda}r}W\del{r} = \colorcancel{-\kappa^2W\del{r}}{red}\\
	\Leftrightarrow& \del{\frac{2\kappa\sigma}r - \frac{\lambda}r}W\del{r} - 2\del{\frac{\sigma}r - \kappa}W'\del{r} - W''\del{r} = 0
\end{align*}

\subsection{}

Nun soll als Ansatz für $W\del{r}$ das Polynom
\begin{align*}
	W\del{r}=\sum_sc_sr^s
\end{align*}
verwendet werden. Einsetzen gibt:
\begin{align*}
	&\del{2\kappa\sigma - \lambda}\sum_sc_sr^{s - 1} - 2\sigma\sum_ssc_sr^{s - 2} - 2\kappa\sum_ssc_sr^{s - 1} - \sum_ss\del{s - 1}c_sr^{s - 2} = 0\\
	\Leftrightarrow&\del{2\kappa\sigma - \lambda}\sum_sc_sr^{s - 1} - 2\sigma\sum_s\del{s + 1}c_{s + 1}r^{s - 1} + 2\kappa\sum_ssc_sr^{s - 1} - \sum_ss\del{s + 1}c_{s + 1}r^{s - 1} = 0\\
	\Leftrightarrow&\sum_s\del{\del{2\kappa\sigma - \lambda + 2\kappa s}c_s - \del{2\sigma\del{s + 1} + s\del{s + 1}}c_{s + 1}}r^{s - 1} = 0\\
	\Leftrightarrow&\sum_s\del{\del{2\kappa\del{\sigma + s} - \lambda}c_s - \del{s + 1}\del{2\sigma + s}c_{s + 1}}r^{s - 1} = 0\\
	\Leftrightarrow&\del{2\kappa\del{\sigma + s} - \lambda}c_s - \del{s + 1}\del{2\sigma + s}c_{s + 1} = 0\\
	\Leftrightarrow&c_{s + 1} = c_s\frac{2\kappa\del{\sigma + s} - \lambda}{\del{s + 1}\del{2\sigma + s}}
\end{align*}
Nun stellt sich die Frage ob die Summe für $r\to\infty$ verschwindet, also ob die Funktion noch normierbar ist. Dazu betrachte ich folgenden Grenzwert:
\begin{align*}
	\lim_{s\to\infty}\frac{c_{s + 1}}{c_s}	&= \lim_{s\to\infty}\frac{2\kappa\del{\sigma + s} - \lambda}{\del{s + 1}\del{2\sigma + s}}\\
											&= \lim_{s\to\infty}\frac{2\kappa\sigma + 2\kappa s - \lambda}{s^2 + s + 2\sigma s + 2\sigma}\\
											&\approx 3\kappa + \frac{\kappa}{\sigma} + \frac{2\kappa s - \lambda}{2\sigma}\\
											&\geq0
\end{align*}
Die Grenzwertbetrachtung zeigt, dass das Polynom für große $s$ nicht verschwindet, sondern die Koeffizienten sogar größer werden. Da dies nicht normierbar ist, muss die Rekursionsformel abbrechen, weswegen gelten muss:
\begin{align*}
	2\kappa\del{\sigma + s} \overset{!}{=} \lambda
\end{align*}
Somit bekommt man die folgende Energie:
\begin{align*}
	&\kappa^2 = \frac{\lambda^2}{4\del{\sigma + s}^2}\\
	\Leftrightarrow& -\frac{2mE}{\hbar^2} = \frac{m^2e^4}{\hbar^4\del{\sigma + s}^2}\\
	\Leftrightarrow& E = \frac{-me^4}{2\hbar^2\del{\sigma + s}^2}\\
\end{align*}

\subsection{}

Nun soll die Näherung $\epsilon<<\del{2l + 1}^2$ benutzt werden, was nichts anderes ist, als die Störung zu Vernachlässigen, woraus sofort $\sigma = l+1$ folgt. Dann ist die Energie:
\begin{align*}
	E	&= \frac{-me^4}{2\hbar^2\del{\sigma + s}^2}\\
		&= \frac{-me^4}{2\hbar^2\del{l + 1 + s}^2}\\
		\intertext{%
			Setzte $n\equiv l + s + 1$:
		}
	E_n	&= \frac{-me^4}{2\hbar^2}\frac1{n^2}
\end{align*}


\end{document}

% vim: spell spelllang=de tw=79
