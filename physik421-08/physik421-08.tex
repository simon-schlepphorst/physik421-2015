\documentclass[11pt, ngerman, fleqn, DIV=15, headinclude]{scrartcl}

\usepackage[bibatend, color]{../header}

\hypersetup{
    pdftitle=
}

\renewcommand{\thesubsection}{\thesection.\alph{subsection}}

\usepackage{units}
\usepackage{listings}
\usepackage{beramono}
\lstset{
    basicstyle=\small\tt
}

%\subject{}
\title{Quantenmechanik, Blatt 8}
%\subtitle{}
\author{
    Frederike Schrödel \and Heike Herr \and Jan Weber \and Simon Schlepphorst
}



\begin{document}

\maketitle
\begin{center}
	\begin{tabular}{l|c|c|c|c|c}
		Aufgabe &1&2&3&4&$\Sigma$\\
		\hline
		Punkte &\quad /12 & \quad /20 & \quad /13 & \quad /7 & \quad /52
	\end{tabular}\\
\end{center}

\section{Kommutatoren}

\subsection{ }

\subsection{ }
\begin{align*}
	\sbr{A,B^n} &= B\sbr{A,B^{n-1}} + \sbr{A,B}B^{n-1}\\
	&= B\del{B\sbr{A,B^{n-2}} + \sbr{A,B}B^{n-2}} + 1\sbr{A,B}B^{n-1}\\
	&= B^2 \sbr{A,B^{n-2}} + B^1 \sbr{A,B}B^{n-2} + B^0 \sbr{A,B}B^{n-1}\\
	&= \sum_{s=0}^{n-1} B^s \sbr{A,B} B^{n-s-1}
\end{align*}

\subsection{ }
Zu zeigen ist die Baker-Cambell-Hausdorff Formel $\eup^A\eup^B =
\eup^{A+B}\eup^{\sbr{A,B}/2}$, wenn $\sbr{\sbr{A,B},A} = \sbr{\sbr{A,B},B} = 0$
gilt.

Sei $F\del t = \eup^{tA}\eup^{tB}$:
\begin{align*}
	\od{F}{t} &= A\eup^{tA}\eup^{tB} + \eup^{tA}B\eup^{tB}\\
	&= A\eup^{tA}\eup^{tB} + \sum_{n=0}^\infty \frac{t^n}{n!}
	A^n B \eup^{tB}\\
	&= A\eup^{tA}\eup^{tB} + \sum_{n=0}^\infty \frac{t^n}{n!}\del{BA^n +
	\sbr{A^n,B}} \eup^{tB}\\
	&= A\eup^{tA}\eup^{tB} + B\eup^{tA}\eup^{tB} + \sum_{n=0}^\infty
	\frac{t^n}{n!} n \sbr{A,B} A^{n-1}\eup^{tB}\\
	&= A\eup^{tA}\eup^{tB} + B\eup^{tA}\eup^{tB} + t\sbr{A,B}\sum_{n=0}^\infty
	\frac{t^{n-1}}{\del{n-1}!} A^{n-1}\eup^{tB}\\
	&= A\eup^{tA}\eup^{tB} + B\eup^{tA}\eup^{tB} +
	t\sbr{A,B}\eup^{tA}\eup^{tB}\\
	&= \del{A + B + t\sbr{A,B}}\eup^{tA}\eup^{tB}
\end{align*}
\begin{align*}
	\implies &\int_0^1 \dif F = \int_0^1 \del{A + B +
		t\sbr{A,B}}\eup^{tA}\eup^{tB} \dif t\\
		\iff &\sbr{\eup^{tA}\eup^{tB}}_0^1 =
		\sbr{\eup^{tA+tB+\frac{t^2}2 \sbr{A,B}}}_0^1\\
		\iff &\eup^A\eup^B = \eup^{A+B}\eup^{\frac12\sbr{A,B}}
\end{align*}

\section{Harmonischer Oszillator in zwei Dimensionen}

\section{Virialtheorem}

\section{Angeregte Zustände des Harmonischen Oszillators}


\end{document}

% vim: spell spelllang=de tw=79
